\documentclass[letterpaper,12pt]{article}
%\pagestyle{empty}
%\usepackage[margin=1in,
%vmargin={22pt,.6in},
%includefoot,includehead]{geometry}

% APA referencing style
\usepackage[utf8]{inputenc}
\usepackage{csquotes}  
\usepackage[english]{babel}
%\DeclareLanguageMapping{english}{english-apa}
%\addbibresource{refs.bib}

% for linguistic examples &  interlinear glosses
\usepackage{linguex}
%to prevent numbered examples to start over each chapter
\usepackage{remreset}
\makeatletter
\@removefromreset{ExNo}{chapter}
\makeatother

%for trees
\usepackage{qtree}	
\usepackage{tree-dvips}

% for inserting/editing image
%The command \graphicspath{ {figures/} } tells LATEX that the images are kept in a folder named "figures" under the current directory.
\usepackage{graphicx}
\graphicspath{ {figures/} }
\usepackage[font=small,skip=5pt]{caption}

% for inserting Chinese characters
\usepackage{CJKutf8}

%for strikethrough
\usepackage[normalem]{ulem}	
\pagestyle{empty}

%\usepackage{setspace}
%\usepackage{lsalike}
%\usepackage{times}
%\usepackage{sectsty}
%\sectionfont{\normalsize}
\usepackage{sectsty}
\allsectionsfont{\normalsize}
%\pagestyle{fancy}
%\headheight 35pt
%\lhead{Journal name, Volume ??, No-??}



%title&name
\title{\large \textbf{Modelling and Evaluation of Artifical Intelligence based Virtual Agents}}
\author{\textit{Anonymous Authors}\\
\textit{Florida State University}}
\date{}


%actually start the paper
\begin{document}
\maketitle

\thispagestyle{empty}
\pagestyle{empty}


\begin{abstract}
\noindent Education technology is adapting to more state the art
technologies such as artificial intelligence (AI). AI in
U.S. Education sector is expected to grow by 47.5 percent by the year 2022 \cite{RefWorks:1}.
Artificial Intelligence and Education (AIED) is promising a new paradigm shift 
in 
the world of education technology. Learning situations are transforming 
everyday, more so with the advent of the COVID-19 pandemic, where a lot 
of learning is happening online. As such there 
is an ever growing need for integrating sophisticated technologies like AI to assist
our learning process. In our research, we are focusing on 
modeling virtual agents with AI to study 
collaborative learning situation. Here, \S 1 we develop an immersive learning enviroment with
OpenSim 3D application server \cite{RefWorks:2}, and model AI enhanced virtual agents in the enviroment, to simulate
a comprehensive real world classroom situation. \S 2 We also build a knowledge 
module to crater to domain specific teaching assistance for different learning situations.
Our models for this novel AI enhanced learning enviroment 
covers all three crucial areas of learning environment modelling, which are,   
design methodologies, system components and range of tools 
available for learners, so that a more realistic intervention 
can be performed 
with our AI learning environment.
\end{abstract}

\section*{Introduction}
%\renewcommand{\baselinestretch}{2}
This is an introduction 

%\parencite{RefWorks:192}

%\S 1 presents blah blah.
%\S 2 offers a  proposal for blah blah.
%\S 3 demonstrates how this analysis can account for blah blah.
%\S 4 offers some brief conclusions, and identifies topics for future study.

\section{Section 1}
This is Section 1.

%Here's an example:

%\exg. mali     zai       xue-xi \\
%		Mary   \textsc{prog} study \\
%		`Mary was/is studying.'

%Explain the example.

\subsection{Section 1.1}
This is Section 1.1.

\section{Section 2}
This is Section 2.

\subsection{Section 2.1}
This is Section 2.1.

\section{Section 3}
This is Section 3.

\subsection{Section 3.1}
This is Section 3.1.

\section{Conclusion}
This is a conclusion.


%next line adds the Bibliography to the contents page
%\addcontentsline{toc}{chapter}{Bibliography}
%uncomment next line to change bibliography name to references
%\renewcommand{\bibname}{References}
%\printbibliography
\bibliographystyle{apalike}
\bibliography{./refs}

\vspace{.25in}
%\noindent \textit{Author contact information:\\
% Sherry Yong Chen: \hspace{.05in} sychen@mit.edu}

\end{document}
